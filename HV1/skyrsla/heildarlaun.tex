\subsection{Heildarlaun karla og kvenna í mismunandi störfum milli ára}
Þegar heildarlaun karla og kvenna í mismunandi starfsgreinum voru borin saman kom í ljós að í hverri einustu starfsgrein var munur á launum kynjanna: meðaltal heildarlauna karla er alltaf hærra.

Fyrstu gögnin sem voru skoðuð geymdu meðaltal yfir allar starfsgreinar, sbr. línuritið á Mynd \ref{fig:heildarhask}, og þar sést að þegar allt er tekið til greina hafa karlar í dag u.þ.b. 130 þús. kr. hærri mánaðartekjur en konur. Á línuritinu sést líka að karlar virðast vera fleiri á vinnumarkaði þ.s. línan sem sýnir meðaltal milli karla og kvenna liggur nær körlunum.

Þetta kom svo sem ekki mikið á óvart þegar horft er til þess að sjómenn, sem eru ein tekjuhæsta (ef ekki tekjuhæsta) stéttin á Íslandi, eru flestir karlkyns, auk þess sem iðnaðarmenn eru flestir karlkyns. 
Það var því hugsanlegt að þetta myndi skekkja niðurstöðurnar eitthvað.

Því var ákveðið að skoða gögn úr mismunandi starfsstéttum en sama hvaða gögn voru skoðuð, alltaf höfðu mennirnir hærri heildarlaun. Í flestum starfsstéttum virðist kynjahlutfallið vera frekar jafnt, nema í fyrrnefndum greinum. Þegar gögn um fólk í skrifstofustörfum voru skoðuð kom niðurstaðan heldur betur á óvart.

Í skrifstofustörfum eru konur mun fleiri en karlar, sbr. Mynd \ref{fig:heildarlaun}. Þetta sést á því að línan sem táknar meðaltalið liggur eiginlega alveg ofan í línunni sem táknar heildarlaun kvenna. Þrátt fyrir það eru heildarlaun karla u.þ.b. 50 þús. kr. hærri. Þetta kemur á óvart þar sem upp hafði komið kenning um að heildarlaunin skekktust oft á tíðum einfaldlega vegna þess að fleiri einstaklingar í viðkomandi starfsstétt væru karlmenn, svo virðist hins vegar ekki vera raunin.

Að lokum var fjölda einstaklinga sem hafa háskólamenntun bætt inn á Mynd \ref{fig:heildarhask} til að sjá hvort einhver fylgni væri milli heildarlauna og menntunarstigs einstaklinga. Á grafinu má sjá að þetta tvennt virðist haldast nokkuð örugglega í hendur, þó svo að karlarnir virðist eiga erfiðara með að rífa sig upp eftir árið 2008. Samt sem áður eru gögnin ofboðslega lík.

Lokakenningin sem gæti skýrt hvers vegna heildarlaun karla eru hærri en kvenna á einhvern einfaldan og rökréttann hátt var þá að karlar hljóta einfaldlega að vinna meira og fá þ.a.l. hærri laun. Annað kom hins vegar í ljós þegar fjöldi greiddra vinnustunda var skoðaður.