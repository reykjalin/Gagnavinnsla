\subsection{Heildarlaun karla og kvenna í mismunandi störfum milli ára}
Við bárum saman heildarlaun karla og kvenna í mismunandi starfsgreinum og í hverri einustu starfsgrein er munur á launum kynjanna: meðaltal heildarlauna karla er alltaf hærra.

Fyrstu gögnin sem við skoðuðum geymdu meðaltal yfir allar starfsgreinar, sbr. línuritið á Mynd [\ref{fig:heildarhask}], og þar sést að þegar allt er tekið til greina hafa karlar í dag u.þ.b. 130 þús. kr. hærri mánaðartekjur en konur. Á línuritinu sést líka að karlar virðast vera fleiri á vinnumarkaði þ.s. línan sem geymir meðaltal milli karla og kvenna liggur nær körlunum.

Þetta kom svo sem ekki mikið á óvart þegar horft er til þess að sjómenn, sem eru ein tekjuhæsta (ef ekki tekjuhæsta) stéttin á Íslandi, eru flestir karlkyns, auk þess sem iðnaðarmenn eru flestir karlkyns. Þetta grunaði okkur að myndi skekkja niðurstöðurnar eitthvað.

Við skoðuðum því gögnin úr mismunandi starfsstéttum en sama hvaða gögn voru skoðuð, alltaf höfðu mennirnir hærri heildarlaun. Í flestum starfsstéttum virðist kynjahlutfallið vera frekar jafnt, nema í fyrrnefndum greinum. Okkur brá hins vegar mikið þegar kom að gögnum um fólk í skrifstofustörfum.

Í skrifstofustörfum eru konur mun fleiri en karlar, sbr. Mynd [\ref{fig:heildarlaun}]. Þetta sést á því að línan sem táknar meðaltalið liggur eiginlega alveg ofan í línunni sem táknar heildarlaun kvenna. Þrátt fyrir það eru heildarlaun karla u.þ.b. 50 þús. kr. hærri. Þetta fannst okkur stórmerkilegt þ.s. við héldum að heildarlaunin skekktust oft á tíðum einfaldlega vegna þess að fleiri einstaklingar í viðkomandi starfsstétt væru karlmenn, svo virðist hins vegar ekki vera raunin.

Að lokum bættum við inn á Mynd [\ref{fig:heildarhask}] fjölda einstaklinga sem hafa háskólamenntun til að sjá hvort einhver fylgni væri milli heildarlauna og menntunarstigs einstaklinga. Á grafinu má sjá að þetta tvennt virðist haldast nokkuð örugglega í hendur, þó svo að karlarnir virðist eiga erfiðara með að rífa sig upp eftir árið 2008. Samt sem áður eru gögnin ofboðslega lík.