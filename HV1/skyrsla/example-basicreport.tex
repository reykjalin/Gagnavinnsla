\documentclass[12pt, git, draft]{rureport}


\begin{document} % this tells the compiler that it is time to make
                 % text to print instead of just getting ready.
\maketitle  % make a title page from the Title, Date, and Author

%\section*{Errata} %%section* avoids putting a number 


\section{Inngangur} % sections break up the document into pieces

Í þessu verkefni áttu nemendur að setja fram rannsóknarspurningar sem þeir gætu svarað með gögnum frá Hagstofu Íslands. Fyrst var spurt, hvað er það sem okkur finnst áhugavert, þar sem við allir eru nemendur við Háskóla Reykjavíkur þá fannst okkur tilvalið að fjalla um  menntun og laun ýmissia starfsstétta.Skoðað var meðaltal heildarlauna starfstétta og eftir þær niðurstöður skoðuðum við meðaltalsvinnustundir á milli kynja milli ára,  hvernig menntunargráða ýmissa aldurshópa skiptist niður á milli ára, skoðum kynjahlutfall háskólagráða innan höfuðborgarsvæðis og utan þess.//
Gögnin sem voru fengin voru þannig að hægt var að svara mörgum spurningum og túlka á margan hátt, aðalega vildum við sjá hvernig menntun og laun dreifðist á milli kynja og svo útfrá þeim spurningum var hægt að spyrja sig að enn fleiri spurningum eins og nefnt er að ofan.
%Í þessu verkefni áttu nemendur að setja fram rannsóknarspurningar sem þeir gætu svarað með gögnum frá Hagstofu Íslands. Fyrst var spurt, hvað er það sem okkur finnst áhugavert, þar sem við allir eru nemendur við Háskóla Reykjavíkur þá fannst okkur tilvalið að fjalla um  menntun og laun ýmissia starfsstétta.// 

%Skoðað var meðaltal heildarlauna starfstétta og eftir þær niðurstöður skoðuðum við meðaltalsvinnustundir á milli kynja milli ára,  hvernig menntunargráða ýmissa aldurshópa skiptist niður á milli ára, skoðum kynjahlutfall háskólagráða innan höfuðborgarsvæðis og utan þess.// 


% Áður en hafist var handa byrjuðum við að setja fram nokkuð opnar spurningar sem snérust að menntun, menntunnar stigum, launum fyrir mismunandi aldurshópa og hvort mætti finna eitthverja fylgni í þessum gögnum.



\section{Framkvæmd}

Þegar búið var að velja hvaða gögn(csv skrár) við myndum koma til að með að nota byrjaði einn hópmeðlimur(K.R.Þ.) að forvinna gögnina í python sem myndi einfalda vinnuna fyrir alla og þar með tíma. Inni í þessari forvinnu var meðal annars settar inn í viðeigandi breytur, prentaður út listi með öllum lykil heitum á breytum(keys). Eftir það settist hópurinn niður og skrifaði spurningar sem okkur fannst áhugaverða og líklegt að það kæmi spennandi niðurstöður úr.//

Forritunnar vinnan gat þá hafist og tóku allir meðlimir hópsins að sér 1-3 spurningar og skrifuðu kóða fyrir þær. Þegar allir voru orðnir sáttir við sínar niðurstöður voru gerð föll fyrir hvern og einn kóða sem skiluðu viðeigandi niðurstöðum og var ein yfirskrá gerð sem er keyrð og birtir allar niðurstöðurnar, í þessu tilfelli línu og súlurit sem birt eru í kafla\ref{nidurstodur} hér að neðan.

\section{Aðferðir}

Aðferðafræðin sem var sett á hópfundi, eftir að einn hóp meðlimur (Team leader Krillvélin) var búinn að forvinna gögin, að allir myndu velja spurningu/ar og vinna úr gögnunum og búa til sinn eigin kóða í Python. Síðan voru gögnin borin saman svo hægt væri að rýna í niðurstöðurnar og svara sem flestum spurningum og hvort væri hægt að draga einhverjar niðurstöður úr þeim niðurstöðum sem fengust.

Uppbygging kóðanna er mjög svipuð þótt hver og einn gerði sinn eiginn kóða, það er byrjað á því að taka inn allar þær 
 
%Sú aðferðar fræði var tekinn í pólinn að allir í hópnum myndu vinna sinn eiginn kóða, en á 3 tíma fresti væru teknar tíu mínótur



\section{Niðurstöður}\label{nidurstodur}
%vladimir
\subsection{Menntunarstig aldurshópa milli ára}
Á mynd [\ref{fig:menntunall}]við getum séð að mest fjöldi af menntað fólk á íslandi eru á bili 30 og 49 ára, næst algengasta aldurs hóp er 50 til 64 ára.
Áhugavert að sjá hvernig fjöldi að mentaðu fólk skiptist á milli kyn, menntun og aldrusflókk. Til dæmis:
\begin{itemize}  
	
	\item Breytinging hjá konum meðgrunnmentun í sem gerðist í  2008 þar sem konur í aldrusflokki 50 - 64 ára eru fleiri en  en 30 - 49 ára eins og sérs á mynd [\ref{fig:menntungrunn}]
	
	\item Það eru mun fleira Karlmen með háskólamentun en konur skv. mynd [\ref{fig:menntunhs}], en það virðist vera fleira og fleira konur sem eru að detta í 30 - 49 ára flokk.
	
	\item Það er minni bil milli aldurs flokkum 30 til 49 og 50 til 69 ára hjá baðum kynum,
	og fljöldi að konum men Starf og framhandsmenntun er mjög svipað á milli aldrusflokka 20 til 24 og 50 til 64 ára, miðað við karla skv mynd [\ref{fig:menntunfram}] 
	
	\item Einhvern hlut að vegna það vantar upplýsingar um menntun hjá fleira Körlum en konum skv mynd [\ref{fig:menntunvantar}]
	
\end{itemize}

%Kristófer
\subsection{Heildarlaun karla og kvenna í mismunandi störfum milli ára}
Þegar heildarlaun karla og kvenna í mismunandi starfsgreinum voru borin saman kom í ljós að í hverri einustu starfsgrein var munur á launum kynjanna: meðaltal heildarlauna karla er alltaf hærra.

Fyrstu gögnin sem voru skoðuð geymdu meðaltal yfir allar starfsgreinar, sbr. línuritið á Mynd \ref{fig:heildarhask}, og þar sést að þegar allt er tekið til greina hafa karlar í dag u.þ.b. 130 þús. kr. hærri mánaðartekjur en konur. Á línuritinu sést líka að karlar virðast vera fleiri á vinnumarkaði þ.s. línan sem sýnir meðaltal milli karla og kvenna liggur nær körlunum.

Þetta kom svo sem ekki mikið á óvart þegar horft er til þess að sjómenn, sem eru ein tekjuhæsta (ef ekki tekjuhæsta) stéttin á Íslandi, eru flestir karlkyns, auk þess sem iðnaðarmenn eru flestir karlkyns. 
Það var því hugsanlegt að þetta myndi skekkja niðurstöðurnar eitthvað.

Því var ákveðið að skoða gögn úr mismunandi starfsstéttum en sama hvaða gögn voru skoðuð, alltaf höfðu mennirnir hærri heildarlaun. Í flestum starfsstéttum virðist kynjahlutfallið vera frekar jafnt, nema í fyrrnefndum greinum. Þegar gögn um fólk í skrifstofustörfum voru skoðuð kom niðurstaðan heldur betur á óvart.

Í skrifstofustörfum eru konur mun fleiri en karlar, sbr. Mynd \ref{fig:heildarlaun}. Þetta sést á því að línan sem táknar meðaltalið liggur eiginlega alveg ofan í línunni sem táknar heildarlaun kvenna. Þrátt fyrir það eru heildarlaun karla u.þ.b. 50 þús. kr. hærri. Þetta kemur á óvart þar sem upp hafði komið kenning um að heildarlaunin skekktust oft á tíðum einfaldlega vegna þess að fleiri einstaklingar í viðkomandi starfsstétt væru karlmenn, svo virðist hins vegar ekki vera raunin.

Að lokum var fjölda einstaklinga sem hafa háskólamenntun bætt inn á Mynd \ref{fig:heildarhask} til að sjá hvort einhver fylgni væri milli heildarlauna og menntunarstigs einstaklinga. Á grafinu má sjá að þetta tvennt virðist haldast nokkuð örugglega í hendur, þó svo að karlarnir virðist eiga erfiðara með að rífa sig upp eftir árið 2008. Samt sem áður eru gögnin ofboðslega lík.

Lokakenningin sem gæti skýrt hvers vegna heildarlaun karla eru hærri en kvenna á einhvern einfaldan og rökréttann hátt var þá að karlar hljóta einfaldlega að vinna meira og fá þ.a.l. hærri laun. Annað kom hins vegar í ljós þegar fjöldi greiddra vinnustunda var skoðaður.

%Róbert
\subsection{Kynjahlutfall vinnustunda milli ára}
Þegar skoðaðar voru greiddar stundir(meðaltal yfir árið) á viku fyrir fulla vinnu þá var skoðað hvort væri mikill munur á greiddum vinnustundum á viku milli kynjana og hvernig það dreifðist yfir árin
Það kom í ljós að þegar um var verið að ræða allar starfstéttir þá sést að mismunur á milli karla og kvenna er 4,2 klukkustundir fyrir greidda vinnustundir í fullri vinnu og árið 2014 er þessi mismunur kominn niður í 2,5 greiddar vinnustundir í fullri vinnu.
Það sést að mismunurinn minnkar til ársins 2002 og eykst svo og eftir árið 2008 það minnkar bilið aftur (sjá mynd: [\ref{fig:unnirtimar1}]). 
\newline
\newline
Þegar við skoðum einungis skrifstofustarf þá sjáum við að munurinn árið 1998 er tæp ein greidd vinnustund, karlar eru með 44,2 vinnustundir á viku á meðan konur vinna 43,3 vinnustundir. Árið 2014 þá munurinn svipaður en það er akkúrat 1 vinnustund sem skilur kynin að.
Hérna er þróunin frekar jöfn en eftir árið 2001 þá eykst munurinn en minnkar aftur eftir árið 2009 (sjá mynd: [\ref{fig:unnirtimar1}]).
\newline
\newline
Þegar er skoðað greiddar vinnustundir á viku hjá sérfræðimenntuðu fólki þá er mismunurinn ekki mikill og er hann mestur rúm ein greidd vinnustund milli ára og árið 2000 þá taka konurnar fram úr karlmönnum og fá fleiri greiddar vinnustundir, en fljótlega eftir 2001 þá fá karlmenn fleiri greiddar vinnustundir það helst svo út 2014 (sjá mynd: [\ref{fig:unnirtimar2}]).
\newline
\newline
Þegar skoðað er greiddar stundir fyrir stjórnendur þá er mismunurinn milli greiddra vinnustunda minna en hálf klukkustund árið 1998 og fljótlega þá fara konur að fá fleiri greiddar vinnustundir á milli ára heldur en karlar. Þegar liðið er á árið 2007 þá fara karlar framúr konum og fá fleiri greiddar vinnustundir en munurinn er innan við eina vinnustund (sjá mynd: [\ref{fig:unnirtimar2}]). 

\begin{figure}
	\centering 
	\includegraphics[width=\textwidth]{../graphics/Haskolamentun_karlar_innan_utan_hs.png}
	\caption{Modular dependency diagram for the electrical circuit \label{fig:menntukarla}}
\end{figure}

\begin{figure}
	\centering 
	\includegraphics[width=\textwidth]{../graphics/Haskolamentun_konur_innan_utan_hs.png}
	\caption{Modular dependency diagram for the electrical circuit \label{fig:menntunkonur}}
\end{figure}

\begin{figure}
	\centering 
	\includegraphics[width=\textwidth]{graphics/heildar_laun.png}
	\caption{Meðaltal heildarlauna karla, kvenna og beggja í skrifstofustarfi \label{fig:heildarlaun}}
\end{figure}

\begin{figure}
	\centering 
	\includegraphics[width=\textwidth]{graphics/heildar_laun_og_haskolamentun.png}
	\caption{Meðaltal heildarlauna karla, kvenna og beggja á vinnumarkaði ásamt háskólamenntun \label{fig:heildarhask}}
\end{figure}

\begin{figure}
	\centering 
	\includegraphics[width=\textwidth]{../graphics/medaltal_stadalfravik_menntun_utan_innan_hs.png}
	\caption{Modular dependency diagram for the electrical circuit \label{fig:stdhs}}
\end{figure}

\begin{figure}
	\centering 
	\includegraphics[width=\textwidth]{../graphics/mentun_aldrusflokkar_alls.png}
	\caption{Aldurflokkar / menntun: Alls \label{fig:menntunall}}
\end{figure}

\begin{figure}
	\centering 
	\includegraphics[width=\textwidth]{../graphics/mentun_aldrusflokkar_grunnmenntun.png}
	\caption{Aldurflokkar / menntun: Grunn- \label{fig:menntungrunn}}
\end{figure}

\begin{figure}
	\centering 
	\includegraphics[width=\textwidth]{../graphics/mentun_aldrusflokkar_haskolamenntun.png}
	\caption{Aldurflokkar / menntun: Háskóla- \label{fig:menntunhs}}
\end{figure}

\begin{figure}
	\centering 
	\includegraphics[width=\textwidth]{../graphics/mentun_aldrusflokkar_starfs_og_framhaldsmenntun.png}
	\caption{Aldurflokkar / menntun: Framhalds- Alls \label{fig:menntunfram}}
\end{figure}

\begin{figure}
	\centering 
	\includegraphics[width=\textwidth]{../graphics/mentun_aldrusflokkar_upplysingar_vantar.png}
	\caption{Aldurflokkar / menntun: Upplýsingar vantar \label{fig:menntunvantar}}
\end{figure}

\begin{figure}
	\centering 
	\includegraphics[width=\textwidth]{../graphics/unnir_timar1.png}
	\caption{Greiddar vinnustundir hjá öllum vinnustéttum og skrifstofufólki \label{fig:unnirtimar1}}
\end{figure}

\begin{figure}
	\centering 
	\includegraphics[width=\textwidth]{../graphics/unnir_timar2.png}
	\caption{Greiddar vinnustundir hjá sérfræðimenntuðu fólki og stjórnendum \label{fig:unnirtimar2}}
\end{figure}

%
\printbibliography
\end{document} % this tells the compiler that we are done

% These are variables for the editor Emacs
%%% Local Variables: 
%%% TeX-command-BibTeX: biber
%%% mode: latex
%%% TeX-master: t
%%% End:
