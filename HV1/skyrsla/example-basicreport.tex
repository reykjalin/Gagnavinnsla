\documentclass[12pt, svn, draft]{rureport}
\svnid{$Id: example-techreport.tex 48 2014-10-23 15:40:52Z foley $}
\svnidlong{$HeadURL: https://projects.cs.ru.is/svn/mechatronics/templates/techreport/example-techreport.tex $}
{$LastChangedDate: 2014-10-23 15:40:52 +0000 (fim., 23 okt. 2014) $}
{$LastChangedRevision: 48 $}
{$LastChangedBy: foley $}
% if you'd like the above information to be updated,
% use svn properties to set svn:keywords to for Id and URL (or HeadURL)
% Don't forget to set the draft to final before submitting

%% The default fixmes are:  \fxnote{} \fxwarning{} \fxerror{} \fxfatal{} (same as \fixme{})
% if you want personalized fixmes, then register the authors here
\FXRegisterAuthor{jf}{jtf}{foley}
% this registers \jfnote{}, \jfwarning{}, \jferror{}, \jffatal{}
% note the use of the two-letters 
% The three-letters are for in a fixme environment

\author{Haukur Hlíðberg, Jón Kr. Helgason, Kristófer Reykjalín, Vladimir Omelianov, Róbert B. Ólafsson}  % My name, for the titlepage
\title{Basic Report Template}  % The title, for the titlepage
%\course{VT 1013 Hönnun}  % pick your class
%\course{T-411-MECH Mechatronics 1}
\instructor{Eyjólfur Ingi Ásgeirsson}
\graphicspath{{graphics/}{Graphics/}{./}}
%% declare the paths(s) where you graphics files can be found

\usepackage[backend=biber, bibencoding=utf8, style=ieee]{biblatex}
%\DeclareLanguageMapping{american}{american-apa}  % needed for style=apa
% If you set backend=bibtex, it will use bibtex for processing (old way)
% if you set backend=biber, you can use UTF8 characters such as Þ and ð but you will have to remember to switch from using bibtex to biber in your client
% See Chapter 7 of http://afs.rnd.ru.is/project/htgaru/trunk/how-to-get-around-projects.pdf 
% If you just want to use bibtex directly without using BibLaTeX, then comment out the line.
\addbibresource{references.bib}

\usepackage[final,hidelinks]{hyperref} % must be last package loaded
% it makes hyper-references (citations, URLs, etc) clickable

\begin{document} % this tells the compiler that it is time to make
                 % text to print instead of just getting ready.
\maketitle  % make a title page from the Title, Date, and Author

%\section*{Errata} %%section* avoids putting a number 


\section{Inngangur} % sections break up the document into pieces
Í þessu verkefni áttu nemendur að setja fram spurningar sem þeir gætu svarað með gögnum frá Hagstofu Íslands. Áður en hafist var handa byrjuðum við að setja fram nokkuð opnar spurningar sem snérust að menntun, menntunnar stigum, launum fyrir mismunandi aldurshópa og hvort mætti finna eitthverja fylgni í þessum gögnum.



\section{Framkvæmd}
Til að byrja með skrifaði Team leader Reykjalín forrit sem tók inn öllgögnin, setti inn í viðeigandi breytur sem allir innan hópsins nutu góðs af en lykilatriðið sem stitti okkur gríðarlegan tíma var að í forritinu var prentaður út listi með öllum lykil nöfnum á breytum(keys). Eftir það settist hópurinn niður og skrifaði spurningar sem okkur fannst áhugaverða og líklegt að það kæmi spennandi niðurstöður úr.
 
Þá hófst forritunnar vinnan, allir meðlimir hópsins tóku að sér 2-3 spurningar og skrifuðu kóða fyrir þær. Þegar allir voru orðnir sáttir við sínar niðurstöður voru gerð föll fyrir hvern og einn kóða sem skiluðu viðeigandi niðurstöðum og ein yfirskrá var gerð sem samnar saman öllum niðurstöðum, í þessu tilfelli línu og súlurit sem birt eru í kafla\ref{nidurstodur} hér að neðan.

\section{Aðferir}
Sú aðferðar fræði var tekinn í pólinn að allir í hópnum myndu vinna sinn eiginn kóða, en við 



\section{Niðurstöður}\label{nidurstodur}




\printbibliography
\end{document} % this tells the compiler that we are done

% These are variables for the editor Emacs
%%% Local Variables: 
%%% TeX-command-BibTeX: biber
%%% mode: latex
%%% TeX-master: t
%%% End:
