\documentclass[12pt, svn, draft]{rureport}
\svnid{$Id: example-techreport.tex 48 2014-10-23 15:40:52Z foley $}
\svnidlong{$HeadURL: https://projects.cs.ru.is/svn/mechatronics/templates/techreport/example-techreport.tex $}
{$LastChangedDate: 2014-10-23 15:40:52 +0000 (fim., 23 okt. 2014) $}
{$LastChangedRevision: 48 $}
{$LastChangedBy: foley $}
% if you'd like the above information to be updated,
% use svn properties to set svn:keywords to for Id and URL (or HeadURL)
% Don't forget to set the draft to final before submitting

%% The default fixmes are:  \fxnote{} \fxwarning{} \fxerror{} \fxfatal{} (same as \fixme{})
% if you want personalized fixmes, then register the authors here
\FXRegisterAuthor{jf}{jtf}{foley}
% this registers \jfnote{}, \jfwarning{}, \jferror{}, \jffatal{}
% note the use of the two-letters 
% The three-letters are for in a fixme environment

\author{Haukur Hlíðberg, Jón Kr. Helgason, Kristófer Reykjalín, Vladimir Omelianov, Róbert B. Ólafsson}  % My name, for the titlepage
\title{Basic Report Template}  % The title, for the titlepage
%\course{VT 1013 Hönnun}  % pick your class
%\course{T-411-MECH Mechatronics 1}
\instructor{Eyjólfur Ingi Ásgeirsson}
\graphicspath{{graphics/}{Graphics/}{./}}
%% declare the paths(s) where you graphics files can be found

\usepackage[backend=biber, bibencoding=utf8, style=ieee]{biblatex}
%\DeclareLanguageMapping{american}{american-apa}  % needed for style=apa
% If you set backend=bibtex, it will use bibtex for processing (old way)
% if you set backend=biber, you can use UTF8 characters such as Þ and ð but you will have to remember to switch from using bibtex to biber in your client
% See Chapter 7 of http://afs.rnd.ru.is/project/htgaru/trunk/how-to-get-around-projects.pdf 
% If you just want to use bibtex directly without using BibLaTeX, then comment out the line.
\addbibresource{references.bib}

\usepackage[final,hidelinks]{hyperref} % must be last package loaded
% it makes hyper-references (citations, URLs, etc) clickable

\begin{document} % this tells the compiler that it is time to make
                 % text to print instead of just getting ready.
\maketitle  % make a title page from the Title, Date, and Author
\listoffixmes{}
%\section*{Errata} %%section* avoids putting a number 


\section{Inngangur} % sections break up the document into pieces
Í þessu verkefni áttu nemendur að setja fram spurningar sem þeir gætu svarað með gögnum frá Hagstofu Íslands. Áður en hafist var handa byrjuðum við að setja fram nokkuð opnar spurningar sem snérust að menntun, menntunnar stigum, launum fyrir mismunandi aldurshópa og hvort mætti finna eitthverja fylgni í þessum gögnum.



\section{Framkvæmd}
Til að byrja með skrifaði Team leader Reykjalín forrit sem tók inn öllgögnin, setti inn í viðeigandi breytur sem allir innan hópsins nutu góðs af en lykilatriðið sem stitti okkur gríðarlegan tíma var að í forritinu var prentaður út listi með öllum lykil nöfnum á breytum(keys)\fxnote{Skýra þetta eh annað ?}. Eftir það settist hópurinn niður og skrifaði spurningar sem okkur fannst áhugaverða og líklegt að það kæmi spennandi niðurstöður úr.

\section{Testing}
What was your testing strategy?  What kinds of tests were performed? Be very specific.  How did your system perform?  Make sure that there is a table/list indicating your system's quantitative performance. As a simple example, refer to Table~\ref{table:testing}



\section{Usage}

Who is the audience for your device?  What knowledge/training does
the person using your device need to use
it?  What further skills would be needed to develop your idea further?

\subsection{Installation}
How does the user install the software/hardware necessary to use your
device?  If they have to download software, make sure that the
locations are here.

If multiple steps are involved, make sure that they are clear and specific.  For example:
\begin{enumerate}
\item Turn on the computer
\item Start windows 7 32-bit or 64-bit
\item Login as a user with administrator privileges
\item Start a web browser
\item Goto \url{http://sourceforge.net/projects/tortoisesvn/files/}
\item Install the appropriate version of TortoiseSVN for your platform.  {\em Do not install version 1.7 if you use Solidworks!}
\item Reboot your computer
\end{enumerate}

\subsection{Instructions}
How does the user use the device/software?  What sorts of commands do
they need to type?  Are there things they should
avoid?\footnote{e.g. don't connect the power to ground or you will
  cause a fire}

If the users need to write additional code to use your system, make
sure you give an example that works\footnote{For larger projects, you
  would create an Application Programming Interface (API).  We may
  cover this later in the term, time willing.}.

\section{Results and Discussion}
In this section you discuss any issues that came up while developing
the system.  If you found something particularly interesting,
difficult, or an important learning experience, put it here.  This is
also a good place to put additional figures and data.



\begin{figure}
\centering
\includegraphics[height=20mm]{ru-logo}
\caption{The logo of Reykjavík University}\label{fig:ru-logo}
\end{figure}
\emph{NOTE: Figures and Tables need to be properly formatted and
  referenced in the text. Number figures/tables consecutively, include
  captions, and refer to the figures/tables in the text
  e.g. Figure~\ref{fig:ru-logo}. Equations need to be numbered
  consecutively as well.  Equations need to have each of their
  variables defined when they are first used or redefined.  If you
  need to refer to particular places in the document, use numbered
  references.  When you put a label{} command }

As an example of an equation, in Equation~\ref{eq:freq} is the relationship between angular frequency and hertz:
\begin{equation}
f = \frac{1}{T} = \frac{1}{2\pi\omega}\label{eq:freq}
\end{equation}
where $f$ is frequency in \si{\hertz}, $T$ is period in \si{\second}, and
$\omega$ is angular frequency in \si{\radian\per\second}.

Note that the \ref{} and the \label{} must match when using \ref{}, it
has to be AFTER the item you are trying to point to or inside of it.
It is often safest to put inside e.g.:\begin{verbatim}\caption{This is
    a caption\label{fig:mycaption}}\end{verbatim}

As an example of referring to a particular part of the document, the
Limitations are in Section~\ref{Limitations} on
page~\pageref{Limitations}

\section{Conclusion}
Summarize the key capabilities of your system. Be specific and state
whether you met your original objectives/requirements. If you could
not, explain why.  How did you compensate so that you could still
complete the system?

Finish the conclusion explaining why customers would care about this
system.  Put it into the context of how it might be used as a product.

\subsection{Future work}
In this section, you explain how you could further develop the
idea/system to do more.  Why would people be interested in these
additions?  Roughly how much work would it take?

\section{References}
List your references (journal articles, textbook pages, etc.). Use the
IEEE citation style, referring to each reference with a consecutive
number in the body of the text and listing the references in the same
numerical order in the References section. Here is a link to a website
describing the IEEE citation style (from the University of
Canterbury):

\url{http://library.canterbury.ac.nz/services/ref/ieee.shtml}

Since you are using \LaTeX{}, you don't need to worry about how it is formatted.
The formatting is taken care of by BibTeX or related tools.  Read Chapter 7 of 
\url{http://afs.rnd.ru.is/project/htgaru/trunk/how-to-get-around-projects.pdf}
The IEEETran.bst format file is included with the template, which is only needed if you are using BibTeX.

As an example of a citation, \cite{carryer2011IntroMechatronics} is
the textbook for T-411-MECH Mechatronics 1.
If you are testing Biber/Biblatex, this citation contains Icelandic characters \cite{foley2013dustcloud}

\section{Appendix}
\subsection{Design documents}
Put CAD drawings, additional sketches

\printbibliography
\end{document} % this tells the compiler that we are done

% These are variables for the editor Emacs
%%% Local Variables: 
%%% TeX-command-BibTeX: biber
%%% mode: latex
%%% TeX-master: t
%%% End:
