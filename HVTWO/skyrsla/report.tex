\documentclass[12pt, git, final]{rureport}


\begin{document} % this tells the compiler that it is time to make
                 % text to print instead of just getting ready.
\maketitle  % make a title page from the Title, Date, and Author

%\fxnote{skoða titil á skýrslu, sbr. forsíðu}

%\section*{Errata} %%section* avoids putting a number 

\section{Inngangur} % sections break up the document into pieces

Hópverkefni 2 gengur út á það að fara inn á síðuna http://grouplens.org/datasets/movielens/ og ná þar í gagnasett sem inniheldur 10 milljón einkunnargjafir, 100 þúsund 'tags' á 10 þúsund myndum frá 72 þúsund notendum.
\newline
Nemendur eiga svo að skrifa python forrit sem tala við postgresqsl sem geymir MovieLens gagnagrunninn. Forritið á að virka á þann hátt að þegar notandi slær inn heiti á kvikmynd þá á einhvern máta á að birtast notandanum viðbótarupplýsingar þar sem stungið er upp á sambærilegum myndum og myndinni sem var slegin inn í upphafi í forritið.
\newline
\newline
Fljótlega og við byrjuðum að hefjast handa þá vildum við útfæra forritið okkur með graphical user interface til að hafa þetta snyrtilegt og eins 'user friendly' og við gátum á þessum stutta tíma sem nemendur höfðu til að klára verkefnið.
\section{Framkvæmd}

blabla

\section{Niðurstöður}\label{nidurstodur}
blabla



\section{Samantekt}

blabla

\pagebreak



%
\clearpage
\printbibliography

\end{document} % this tells the compiler that we are done

% These are variables for the editor Emacs
%%% Local Variables: 
%%% TeX-command-BibTeX: biber
%%% mode: latex
%%% TeX-master: t
%%% End:
