% custom.tex gets loaded if it exists by rubeamer.cls

% Presentation information
\title{Lokaverkefni - War And Eclipses}
\subtitle{Final Presentation}
\author[H.H., J.K.H., K.R.Þ., R.B.Ó., V.O]{Haukur Hlíðberg, Jón Kr. Helgason, Kristófer R. Þorláksson, Róbert B. Ólafsson, Vladimir Omelianov}
\institute[RU]{
  Department of Science and Engineering (TVD) \\
  Reykjavík University}
\date{16 December, 2015\\ Rev~\svnrev{}} %% Put the real presentation day so it doesn't
                        %% change later
\graphicspath{{graphics/}{graphics}}
\newcommand{\bookcite}{\parencite{carryer2011IntroMechatronics}\xspace}
\newcommand{\cg}[2][0.8]{\centering\includegraphics[width=#1\textwidth,height=#1\textheight,keepaspectratio]{#2}}
\newcommand{\bookcap}[1]{#1\bookcite}
\newcommandx{\bookframe}[3][3=]{\begin{frame}{#2}
\cg{#1}\label{#1}\\#3%\bookcite{}
\end{frame}}
\newcommandx{\bookfig}[3][3=]{\begin{frame}{#2}
\begin{figure}\cg{#1}\\caption{#3\bookcite{}\label{fig:#1}}\end{figure}\label{#1}
\end{frame}}
\newcommandx{\booktab}[3][3=]{\begin{frame}{#2}
\begin{table}\cg{#1}\caption{#3\bookcite{}\label{tab:#1}}\end{table}\label{#1}
\end{frame}}
\newcommand{\twocol}[2]{
\begin{columns}
    \begin{column}{.5\linewidth}
    #1
        \end{column}
        \begin{column}{.5\linewidth}
        #2
        \end{column}
\end{columns}
}
%% Commonly used notation/abbrevs, converted into latex commands:
\renewcommand{\d}[1]{\;\textsf{d}#1}
\newcommand{\pd}[1]{\partial #1}
\newcommand{\D}{\;\textsf{D}}
\providecommand{\e}[1]{\ensuremath{\times 10^{#1}}}


%% Tables that resize
\usepackage{tabularx}

\newcolumntype{Y}{>{\raggedright\arraybackslash}X}

%% Better table styling
\usepackage{booktabs}

%% lots of math power
\usepackage{amsmath}

%% More math magic
%% Particularly \begin{Bmatrix*}[l] which left-justifies
\usepackage{mathtools}

%% Useful symbols like \texteuro
\usepackage{textcomp}

%% Chemical formulas
% % Disabling by default because LaTeX 3 changes can cause problems for some MikTeX users
%\usepackage[version=3]{mhchem}

%% SI units we use a lot
\newcommand{\SIgKwh}[1]{\SI{#1}{\gram\per\kilo\watt\per\hour}} 
\newcommand{\SIgpm}[1]{\SI{#1}{\gram\per\minute}}
\newcommand{\SINm}[1]{\SI{#1}{\newton\meter}}
\newcommand{\SIcm}[1]{\SI{#1}{\centi\meter}}
\newcommand{\SImm}[1]{\SI{#1}{\milli\meter}}
\newcommand{\SIV}[1]{\SI{#1}{\volt}}
\newcommand{\SIMhz}[1]{\SI{#1}{\mega\hertz}}
\newcommand{\SIlpm}[1]{\SI{#1}{\liter\per\hour}}
\newcommand{\SIdegC}[1]{\SI{#1}{\degreeCelsius}}
%% Engines use cc for cubic-centimeter
%% which looks different tnan \centi\meter\cubed
%% This is overkill, but it maintains typesetting consistency
%\newunit{\dispcc}{cc}
%\newcommand{\SIcc}[1]{\SI{#1}{\dispcc}}
%% And apparently \newunit has been removed from my 2013 copy of siunitx, how annoying.
\newcommand{\SIcc}[1]{#1~cc}
\newcommand\CA[1]{\ensuremath{\text{CA}_{#1}}}
\newcommand\FR[1]{\ensuremath{\text{FR}_{#1}}}
\newcommand\DP[1]{\ensuremath{\text{DP}_{#1}}}
\newcommand\C[1]{\ensuremath{\text{C}_{#1}}}

%%% Local Variables:
%%% mode: latex
%%% TeX-master: "icad2015-berthor_foley-uget"
%%% End:
%% Extremely common abbreviations
\newcommand{\kg}[1]{\SI{#1}{\kilo\gram}}
\newcommand{\mA}[1]{\SI{#1}{\milli\ampere}}
\newcommand{\amp}[1]{SI{#1}{\ampere\xspace}}


\newcommand{\realset}{\ensuremath{\mathbb{R}}}
\newcommand{\intset}{\ensuremath{\mathbb{Z}}}
\newcommand{\ratset}{\ensuremath{\mathbb{Q}}}


\newcommand{\adfr}{\ensuremath{\text{FR}}}
\newcommand{\addp}{\ensuremath{\text{DP}}}
\newcommand{\adpv}{\ensuremath{\text{PV}}}

\newcommand{\bibinit}[1][references]{
% for bibtex, just comment out the line
\addbibresource{#1.bib} % biber/biblatex
}
 
\newcommand{\bibframe}[1][references]{%
  \begin{frame}[allowframebreaks]{References}
    %% bibtex
    % \bibliographystyle{apalike}
    % \bibliography{#1}
    
    %% biber/biblatex.  Database at beginning of file
    \printbibliography
  \end{frame}
}

\newcommand{\titleframe}{%
  \begin{frame}[plain]
    \titlepage
  \end{frame}
}

\tikzstyle{startstop} = [rectangle, rounded corners, minimum width=2cm, minimum height=1cm,text centered, draw=black, fill=red!30]
\tikzstyle{io} = [trapezium, trapezium left angle=70, trapezium right angle=110, minimum width=3cm, minimum height=1cm, text centered, draw=black, fill=blue!30]
\tikzstyle{process} = [rectangle, rounded corners, minimum width=2cm, minimum height=1cm,text centered, draw=black, fill=blue!30]
\tikzstyle{decision} = [diamond, minimum width=3cm, minimum height=1cm, text centered, draw=black, fill=green!30]
\tikzstyle{arrow} = [thick,->,>=stealth]

\usepackage{animate}

%%% Local Variables:
%%% mode: latex
%%% TeX-master: t
%%% End:
