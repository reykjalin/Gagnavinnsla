\documentclass[12pt, git, final]{rureport}


\begin{document} % this tells the compiler that it is time to make
                 % text to print instead of just getting ready.
\maketitle  % make a title page from the Title, Date, and Author

%\fxnote{skoða titil á skýrslu, sbr. forsíðu}

%\section*{Errata} %%section* avoids putting a number 
%
\section{Inngangur} % sections break up the document into pieces
Himingeimurinn er fyrir mörgum jafn áhugaverður og hann er að stærð og menn hafa öldum saman spáð mikið í himintunglum og önnur fyrirbæri út í geimi. Ákveðið var því að ná í gögn fyrir sólmyrkva jarðar, ásamt því að ná í gögn fyrir stríð  sem herjað hefur jörðina og vopnasölu út um heim allan. Fyrst og fremst vildi hópurinn skoða hvernig stríð dreifist á jörðina, hvaða vopnasalar eru að selja átakalöndunum, hversu oft og hvar lenda sólmyrkvarnari meðan stríðin geisa og hvort það sé möguleiki á fylgni milli fjölda sólmyrkva og fjölda stríðsátaka? Þessir útgangspunktar verða skoðaðir nánar og gefin skil á niðurstöðunum. 
\section{Framkvæmd}
\subsection{Gögn}
Fyrst var farið á veraldarvefin og leitað af gögnum sem hægt væri að vinna með, gögnin voru fundin á síðu á síðu hjá NASA\cite{Eclipse}, Stockholm international peace research institution\cite{weapon} og Háskólanum í Uppsölum í Svíþjóð\cite{conflict}. Gögnin voru tekin inn og unnið með þau í þremur mismunandi skrám.
\subsection{Hönnun}
Við notumst við Basemap sem er viðbótarpakki fyrir Matplotlib til að sýna hvernig sólmyrkvinn lendir á jörðinni. Gögnin um sólmyrkvan frá NASA eru notaðar til að sýna staðsetningarnar sólmyrkvana á jörðinni, það er notast við dagsetningar og tímasetningar þar sem sólmyrkvinn er sem mestur. 
\subsection{Aðferð}
%
\section{Niðurstöður}\label{nidurstodur}
Sjá má frá gögnunum að því oftar sem sólmyrkvi skellur á  jörðinni þá eru meiri/minni líkur á að: BLA



\pagebreak


\clearpage
\printbibliography

\end{document} % this tells the compiler that we are done

% These are variables for the editor Emacs
%%% Local Variables: 
%%% TeX-command-BibTeX: biber
%%% mode: latex
%%% TeX-master: t
%%% End:
