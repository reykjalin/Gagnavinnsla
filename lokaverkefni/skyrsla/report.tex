\documentclass[12pt, git, final]{rureport}


\begin{document} % this tells the compiler that it is time to make
                 % text to print instead of just getting ready.
\maketitle  % make a title page from the Title, Date, and Author

%\fxnote{skoða titil á skýrslu, sbr. forsíðu}

%\section*{Errata} %%section* avoids putting a number 
%
\section{Inngangur} % sections break up the document into pieces
Himingeimurinn er fyrir mörgum jafn áhugaverður og hann er að stærð og menn hafa öldum saman spáð mikið í himintunglum og önnur fyrirbæri út í geimi. Ákveðið var að ná í gögn fyrir sólmyrkva sem hafa herjað á jörðna, ásamt því að ná í gögn fyrir stríð og vopnasölu út um allan heim. Fyrst og fremst er skoðað hvernig stríð dreifist á jörðina, hvaða vopnasalar eru að selja átakalöndunum og hvort það sé möguleiki á fylgni milli sólmyrkva og fjölda stríðsátaka
\section{Framkvæmd}
\subsection{Gögn}
Fyrst var farið á veraldarvefin og leitað af gögnum sem hægt væri að vinna með, gögnin voru fundin á síðu á síðu hjá NASA, Stockholm international peace research institution og Háskólanum í Uppsölum í Svíþjóð. Gögnin voru tekin inn og unnið með þau í þremur mismunandi skrám.
\subsection{Hönnun}
Við notumst við Basemap sem er viðbótarpakki fyrir Matplotlib til að sýna hvernig sólmyrkvinn lendir á jörðinni. Hv
\section{Niðurstöður}\label{nidurstodur}

\pagebreak


\clearpage
\printbibliography

\end{document} % this tells the compiler that we are done

% These are variables for the editor Emacs
%%% Local Variables: 
%%% TeX-command-BibTeX: biber
%%% mode: latex
%%% TeX-master: t
%%% End:
